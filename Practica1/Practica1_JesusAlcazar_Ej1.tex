\documentclass[11pt]{article}
    \title{\textbf{Práctica 1 TALF}}
    \author{Jesús Alcázar Pérez 2ºA Ingeniería Informática}
    \date{}

    \addtolength{\topmargin}{-3,5cm}
    \addtolength{\textheight}{4cm}
    %\usepackage[spanish]{babel}
    \usepackage[a4paper, margin=2cm, top=5mm, bottom=15mm]{geometry}

    % Paquetes necesarios
	\usepackage[utf8]{inputenc}
	\usepackage{amsthm, amsmath}
	\usepackage{nccmath} %Para centrar ecuaciones
	\usepackage{graphicx}
	\usepackage{enumitem}

\begin{document}


\maketitle
\thispagestyle{empty}

\section{Find the power set $R^3$ of $R$ = \{(1, 1),(1, 2),(2, 3),(3, 4)\}. Check your answer with the script powerrelation.m and write a \LaTeX\ document with the solution step by step.}

Siendo $R$ = \{(1, 1), (1, 2), (2, 3), (3, 4)\} \\
$R^2$ = $R$ × $R$ = \{(1, 1),(1, 2),(2, 3),(3, 4)\} × \{(1, 1),(1, 2),(2, 3),(3, 4)\} = \\
\{(1, 1),(1, 2),(1, 3),(2, 4)\} \\
$R^3$ = $R$ × $R^2$ = \{(1, 1),(1, 2),(2, 3),(3, 4)\} × \{(1, 1),(1, 2),(1, 3),(2, 4)\} = \\
\{(1, 1),(1, 2),(1, 3),(1, 4)\} \\
Por tanto, $R^3$ = \{(1, 1), (1, 2), (2, 3), (1, 4)\}
\\
\\
Comprobación en Octave: \\
octave:1 powerrelation({['1', '1'], ['1', '2'], ['2', '3'], ['3', '4']}, 3)\\
ans = \{ [1,1] = 11, [1,2] = 12, [1,3] = 13, [1,4] = 14 \}

\end{document}
